\sectionTitle{Selected Projects}{}
\begin{projects}
	\project
	{Generating Task-Aware Scene Descriptions}{Aug'18 - Present}
	{\textit{Extension of work done at University College London} \href{https://usercontext.github.io/SceneTask/}{Project page}}
	{\begin{itemize}
     \item Leveraging \href{https://usercontext.github.io/TaskHierarchy138K/}{TaskHierarchy138K} which aids in the scene description generation by injecting task coherence scores into the state-of-the-art models. This helps the machines to be task-aware and produce more meaningful descriptions.
     \item Work accepted at AAAI 2019 Students Abstracts Program.
     \end{itemize}}

\end{projects}
\begin{projects}
     \project
	{Analytics Sensor Test-Kit for Advanced Process Control}{Nov'17 - May'18}
	{\textit{Internet of Things with \href{http://universe.bits-pilani.ac.in/goa/nitinn/profile}{Nitin Sharma} | Bill \& Melinda Gates Foundation}}
	{\begin{itemize}
     \item Designed a proof-of-concept IoT scalable interface to gather data from sensors installed at the waste water treatment site. This work was in collaboration with Ohio University, Cranfield University, and Caltech.
     \end{itemize}}
    \project
	{Study and Implementation of Critical Protocol Analysis}{Jan'17 - May'17}
	{\textit{Network Measurements with \href{https://prasadtalasila.wordpress.com/}{Prasad Talasila}}}
	{\begin{itemize}
     \item Spawned a cross-collaboration workbench for BITS Darshini, a modular, concurrent, customizable protocol analyzer. This is implemented for sharing the experiments among researchers and experts (for review) in order to save resources on analysis of large pcap files involved in the process.
     \item Modularized and refactored BackboneJS based client module. Decoupled the persistence code from the rest of the codebase. Also developed Vagrant container for easy installation.
     \end{itemize}}
     
     \project
	{Approximate Frontalization of Unconstrained Faces using BEGAN}{Aug'17}
	{\textit{Generative Adversarial Networks with \href{http://www.bits-pilani.ac.in/goa/tirtharaj/profile}{Tirtharaj Dash}}}
	{\begin{itemize}
      \item Boundary Equilibrium GANs give the best results in creation of faces, owing to auto-encoder based GANs. Its interpolatory feature is able to generate a face out of two different faces.
     \item Devised a face-frontalize algorithm which used interpolation feature on mirrored side pose.
     \end{itemize}}
     \project
	{REST API: Custom Forms \& GraphQL replacement}{Dec'16  - Mar'17}
	{\textit{With \href{https://www.linkedin.com/in/dylanwh}{Dylan Hardison}, \href{https://www.linkedin.com/in/cameron-dawson-a9a41a}{Cameron Dawson} \& \href{https://in.linkedin.com/in/wrlach}{William Lachance} (Mozilla)}}
	{\begin{itemize}
      \item Made a custom form example application which uses REST API alongwith auth delegation callbacks to file the application as bugs on bugzilla.mozilla.org. This is to track the data generated through forms as bugs.
     \item Reponsible for introducing the integration of GraphQL based APIs on Mozilla's automation tool, Treeherder. GraphQL queries only the required variables from the database, which improves the efficiency drastically. 
     \end{itemize}}
     \project
	{AutolabCLI: Command Line Interface for AutolabJS}{Aug'16 - Dec'16}
	{\textit{CLI Application with \href{https://prasadtalasila.wordpress.com/}{Prasad Talasila}}}
	{\begin{itemize}
      \item Developed a CLI for AutolabJS which helps in committing the code to local gitlab server, submitting the code from gitlab server to the evaluation nodes and retrieving the results using a socket connection.
     \item Published on npm and currently in use for programming labs at the university.
     \end{itemize}}
\end{projects}    
\vspace{-3mm}

