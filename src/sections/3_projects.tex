\sectionTitle{Select Research Projects}{}
\begin{projects}

\project
	{Interactive Neural Machine Translation (INMT)}{Jan'19 - Present}
	{
	    \textit{Advisors:  \href{https://www.microsoft.com/en-us/research/people/kalikab/}{Dr. Kalika Bali}, \href{https://www.microsoft.com/en-us/research/people/monojitc/}{Dr. Monojit Choudhury}, \href{https://www.microsoft.com/en-us/research/people/monojitc/}{Dr. Sandipan Dandapat}, \href{https://www.microsoft.com/en-us/research/people/monojitc/}{Tanuja Ganu}}
	}
	{\begin{itemize}
	\setlength\itemsep{0.3em}
     \item Worked on understanding how translators can be assisted with suggestions from a machine translation system. On basis of the insights gathered, developed an interactive translation interface to make the translation process quicker and better in terms of quality. ~ [\href{https://microsoft.github.io/inmt/}{\small{\websiteSymbol}}] ~ {\small{\lbrack\textbf{{Demo@EMNLP'19}}\rbrack}}
     \item Engaging with non-profits \href{https://translatorswithoutborders.org/}{Translators without Borders}, Pratham Books' \href{https://storyweaver.org.in/}{Story Weaver} and \href{http://cgnetswara.org/}{CGNet Swara} (covered by \href{https://www.livemint.com/mint-lounge/features/now-a-unique-machine-translation-tool-from-hindi-to-gondi-11597386377981.html}{LiveMint}) looking at possible solutions for deploying INMT for low resource languages. ~ {\small{\lbrack\textbf{{In Submission}}\rbrack}}
     \item Developing new interfaces to tailor to specific use-cases of translation such as document translation and web-page localization (using \href{github.com/microsoft/inmt-browser}{browser extension}) and offline translation (\href{https://github.com/microsoft/INMT-lite}{INMT lite}).
     \end{itemize}}
     
\project
	{State of Language Technologies for Low-Resource Languages}{Jan'19 - Present}
	{
	     \textit{Advisors:  \href{https://www.microsoft.com/en-us/research/people/kalikab/}{Dr. Kalika Bali}, \href{https://www.microsoft.com/en-us/research/people/monojitc/}{Dr. Monojit Choudhury}}
	}
	{\begin{itemize}
	\setlength\itemsep{0.3em}
     \item Conducted a quantitative analysis on disparity of language resources being used at NLP conferences. Created data model of publications (authors, papers, institutions) and used statistical measures as well as entity embeddings to classify and track the progress of representation of different languages over conference iterations. ~ [\href{https://microsoft.github.io/linguisticdiversity/}{\small{\websiteSymbol}}]  ~  {\small{\lbrack\textbf{{ACL'20}}\rbrack}}
     \item Understand and track the impact of different language-based technological interventions that were carried out by \href{http://cgnetswara.org/}{CGNet Swara} in the Gond Community of Chattisgarh. \href{https://en.wikipedia.org/wiki/Gondi_language}{Gondi} is a language spoken by 3 million people however is a severely low-resourced language mainly attributed to non-existence of its own script. {\small{\lbrack\textbf{{LREC'20}}\rbrack}} ~ {\small{\lbrack\textbf{{LT4All}}\rbrack}}
     \item Surveyed the challenges faced for deployment of language technologies to marginalized communities. {\small{\lbrack\textbf{{ICON'19}}\rbrack}}
     \item Coverage/Mentions - {ACL'20}: \href{https://qz.com/1920191/internet-translation-access-creates-a-powerful-digital-divide/}{Quartz},  \href{http://newsletter.ruder.io/issues/reviewing-taking-stock-theme-papers-poisoning-and-stealing-models-multimodal-generation-240687}{NLP Newsletter}, \href{https://ruder.io/nlp-beyond-english/}{ruder.io/nlp-beyond-english}, \href{https://www.underratedml.com/episodes/episode-05-language-independence-and-material-properties}{Underrated ML Podcast}, \href{https://lacunafund.org/language/}{Lacuna Fund}, \href{https://sigtyp.github.io/sigtyp-newsletter-Apr-2020.html}{SIGTYP Newsletter}; {LREC'20}: \href{http://toi.in/HcX74b/a31g}{Times of India}, \href{https://www.hindustantimes.com/india-news/gonds-in-chhattisgarh-get-app-for-news-in-their-language/story-uQEDqDGBIPty7rMCNskTsK.html}{Hindustan Times}, \href{https://www.etvbharat.com/hindi/chhattisgarh/state/raipur/now-tribal-can-hear-news-and-stories-in-their-language/ct20190802191729141}{ETV}
     \end{itemize}}
    
\end{projects}
\begin{projects}
\project
    {Behaviour of Transformer Language Model Attention Heads for Different Stimuli}{Aug'20 - Present}
	{ \textit{Advisor: \href{https://www.microsoft.com/en-us/research/people/monojitc/}{Dr. Monojit Choudhury}}}
	{
	\begin{itemize}
	\setlength\itemsep{0.3em}
     \item Analyzed how the attention heads of BERT change with fine-tuning. Here, we introduce Code-Mixing to the models and judge the \textit{responsivity} of different heads to them being provided as input. {\small{\lbrack\textbf{{AdaptNLP@EACL'21}}\rbrack}}
     \item Analyzing how different attention heads behave to stimuli that is based on a diverse set of linguistic tasks provided as input to BERT. {\small{\lbrack{{Working Paper}}\rbrack} }
     \end{itemize}
     }
     
\project
    {User Tasks and Needs Understanding}{Jan'18 - Present}
	{ \textit{\textit{Advisors: \href{https://sites.google.com/site/emineyilmaz/}{Prof. Emine Yilmaz}, \href{http://rishabhmehrotra.com/}{Dr. Rishabh Mehrotra}, \href{https://bizfaculty.nus.edu.sg/faculty-details/?profId=588}{Prof. Prasanta Bhattacharya}}}}
	{
	\begin{itemize}
	\setlength\itemsep{0.3em}
     \item Designed an ontology of common tasks carried out on a day-to-day basis. These tasks were derived from Wikihow which is collection of ``How-To" questions that can be modelled as users seeking answers on how to solve a particular task.
     \item Showed the use of such a task ontology to produce more coherent image captioning. {\small{\lbrack\textbf{{Abstract@AAAI'19}}\rbrack}}
     \item Analyzed which tasks are of interest to different communities around the world. {\small{\lbrack{{Working Paper}}\rbrack}}
     
     \end{itemize}
     }
\end{projects}    
\vspace{-3mm}

